%%%%%%%%%%%%%%%%%%%%%%%%%%%%%%%%%%%%%%%%%
% University/School Laboratory Report
% LaTeX Template
% Version 2.0 (4/12/12)
%
% This template has been downloaded from:
% http://www.latextemplates.com
%
% License:
% CC BY-NC-SA 3.0 (http://creativecommons.org/licenses/by-nc-sa/3.0/)
%
% Original header:
%
% This is a LaTeX version of the sample laboratory report
% from Virginia Tech's copyrighted 08-09 CHEM 1045/1046 lab manual.
% Reproduction of this one appendix section for academic purposes
% should fall under fair use.
%
%%%%%%%%%%%%%%%%%%%%%%%%%%%%%%%%%%%%%%%%%

%----------------------------------------------------------------------------------------
%	DOCUMENT CONFIGURATIONS
%----------------------------------------------------------------------------------------

\documentclass{article}

% Package to generate and customize Algorithm as per ACM style
\usepackage[ruled]{algorithm2e}
\renewcommand{\algorithmcfname}{ALGORITHM}
\SetAlFnt{\small}
\SetAlCapFnt{\small}
\SetAlCapNameFnt{\small}
\SetAlCapHSkip{0pt}
\IncMargin{-\parindent}

\usepackage{graphicx, enumerate} % Allows the inclusion of images
\usepackage[margin=1in]{geometry}
\title{Unit Project Report \\ CS 425 Compiler Construction} % Title

\author{Lingyi Liu (liu187), Kuan-Yu Tseng (ktseng2)} % Author name

\date{\today} % Specify a date for the report

\begin{document}

\maketitle % Insert the title, author and date

%----------------------------------------------------------------------------------------
%	SECTION 1
%----------------------------------------------------------------------------------------

\section{Chaitin-Briggs Algorithm}
Chaitin-Briggs algorithm\cite{Chaitin:1982} is a register allocation algorithm that utilizes graph coloring on the interference graph that are derived from the live ranges of registers, to allocate physical register for each virtual register. There are 4 major steps for this algorithm:\textbf{Live Range Computation}, \textbf{Interference Graph Construction}, \textbf{Spill Cost Calculation}, \textbf{Graph Coloring}. These steps are described in detail in the following.
\subsection{Live Range Computation}
The Global Live Ranges of a virtual register \emph{vreg} is a partition of the references (definitions or uses) of \emph{vreg}. If one definition \emph{def} of \emph{vreg} is in the Live Range \emph{lr}, then all uses reachable from \emph{def} are also in \emph{lr}. If one use of \emph{vreg} is in the Live Range \emph{lr}, then all defs that reaches the use are also in \emph{lr}. Using the Live Variable Analysis, we can compute the def-use chain for each virtual register. And therefore we can compute the live ranges with \texttt{union-find} algorithm.
\subsection{Interference Graph Construction}
A register \emph{vreg1} is said to interfere with another register \emph{vreg2} if \emph{vreg1} is defined when \emph{vreg2} is live. Therefore, to show the interference among all registers, we can construct a interfernce graph where each node represents a register and each undirected edge represents whether the nodes on both ends interfere with each other. The algorithm for constructing the interfence graph is as follows.
\subsection{Spill Cost Calculation}

\subsection{Graph Coloring}
%----------------------------------------------------------------------------------------
%	SECTION 2
%----------------------------------------------------------------------------------------

\section{Heuristic Approach for Spilling}

%----------------------------------------------------------------------------------------
%	SECTION 3
%----------------------------------------------------------------------------------------

\section{Project Status}
\subsection{What is Working?}
\subsection{What is Not Working?}
\subsection{Potential Improvement}
%----------------------------------------------------------------------------------------
%	SECTION 4
%----------------------------------------------------------------------------------------

\section{Experimetnal Results}
\subsection{Benchmark Programs}
\subsection{Execution Time}
\subsection{Number of Sills}
\bibliographystyle{IEEETran}
\bibliography{lab_report_1}
\end{document}